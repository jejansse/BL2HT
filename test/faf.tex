% $Id: faf.tex,v 1.2 2008/01/02 19:11:53 dvermeir Exp $
%%%%%%%%%%%%%%%%%%%%%%%%%%%%%%%%%%%%%%%%%%%%%%%%%
% \documentclass{ipmu2008}

\documentclass[12pt,a4paper]{article}
% \documentclass[a4paper]{article}
\usepackage{url}

\usepackage{color}
% \usepackage[pdftex]{graphicx}
\usepackage{latexsym}
\usepackage{amsfonts}
\usepackage{amssymb} % \curlywedge, \curlyvee
\DeclareSymbolFontAlphabet{\mathbb}{AMSb}
\newcommand{\Nat}{\ensuremath{\mathbb N}}
\newcommand{\Real}{\ensuremath{\mathbb R}}
\newcommand{\Complex}{\ensuremath{\mathbb C}}
% Some auxiliary symbols for \Forall, \Exists
\DeclareMathSymbol{\FORALL}   {\mathord}{symbols}{"38}
\DeclareMathSymbol{\EXISTS}   {\mathord}{symbols}{"39}
\DeclareMathSymbol{\SUCHTHAT} {\mathbin}{symbols}{"01}
% E.g. \Forall{x\in\Nat}{even(x)\lor odd(x)}
% E.g. \Exists{x\in\Nat}{\not\Exists{y\in\Nat}{y>x}}
% \def\Forall#1#2{{\FORALL#1} \SUCHTHAT #2}
% \def\Exists#1#2{{\EXISTS#1} \SUCHTHAT #2}
\newcommand{\Exists}[2]{\ensuremath{\EXISTS{#1} \SUCHTHAT {#2}}}
\newcommand{\Forall}[2]{\ensuremath{\FORALL{#1} \SUCHTHAT {#2}}}
\def\OExists#1#2{{{\EXISTS^\omega}#1} \SUCHTHAT #2}

\newcommand{\todo}[1]{(\textsc{todo}: \textbf{#1})}
\newenvironment{ltodo}{\begin{description}\item[\textbf{TODO}]}{\end{description}}
\newcommand{\mc}[1]{\ensuremath{\mathcal{#1}}}

% FAF
\newcommand{\abfaf}{\textsc{faf}}
\newcommand{\combine}{\mc{C}}
\newcommand{\combiner}[1]{\ensuremath{\combine_{#1}}}
\newcommand{\J}{\ensuremath{\mc{J}}}
\newcommand{\JJ}[1]{\ensuremath{\mc{J}_{#1}}}
\newcommand{\D}{\ensuremath{\mc{D}}}
\newcommand{\DD}[1]{\ensuremath{\mc{D}_{#1}}}
\newcommand{\JD}[1]{\ensuremath{(\JJ{#1},\DD{#1})}}

% Argumentation frameworks
\newcommand{\attack}{\ensuremath{\not\rightarrow}}
\newcommand{\eattack}{\ensuremath{\attack^e}}
\newcommand{\support}{\ensuremath{\rightarrow}}
\newcommand{\af}[2]{\ensuremath{\langle{#1},{#2}\rangle}}
\newcommand{\baf}[3]{\ensuremath{\langle{#1},{#2},{#3}\rangle}}
\newcommand{\abaf}{\textsc{af}}
\newcommand{\abbaf}{\textsc{baf}}

% Fuzzy
\newcommand{\Fuzzy}[1]{\ensuremath{\mc{F}({#1})}}
\newcommand{\ssupport}[1]{\ensuremath{\mathit{supp}({#1})}}
%\newcommand{\ssupport}[1]{\ensuremath{\bar{#1}}}
\newcommand{\Aggr}[1]{\ensuremath{\mc{A}_{#1}}}
\newcommand{\Negator}{\mc{N}}
\newcommand{\Implicator}{\mc{I}}
\newcommand{\Bi}[1]{\ensuremath{\mc{B}#1}}
\newcommand{\Lattice}{\ensuremath{\mc{L}}}
\newcommand{\fand}{\ensuremath{\curlywedge}}
\newcommand{\for}{\ensuremath{\curlyvee}}
\newcommand{\fneg}{\ensuremath{\sim}}
\newcommand{\fullimg}[2]{\ensuremath{{}^{#2}{#1}}}
\newcommand{\subdirectimg}[2]{\ensuremath{{}_{#2}{#1}}}

\begin{document}

\newcommand{\set}[1]{{\ensuremath{\{ #1 \}}}}

\newcommand{\abfasp}{\textsc{fasp}}
% For logic programs.
% E.g.
%  \begin{program}
%  srule{a}{b,c} % a <- b,c
%  srule{b}{d,c} % b <- d,c
%  \end{program}
\newenvironment{program}{\[\begin{array}{r@{\:\gets\:}l}}{\end{array}\]}
\newenvironment{program2c}{\[\begin{array}{r@{\:\gets\:}l@{\:\:\:}r@{\:\gets\:}l}}{\end{array}\]}
\newenvironment{program3c}{\[\begin{array}{r@{\:\gets\:}lr@{\:\gets\:}lr@{\:\gets\:}l}}{\end{array}\]}
\newcommand{\srule}[2]{\ensuremath{#1 & #2}}
\newcommand{\prule}[2]{\ensuremath{#1\leftarrow#2}}
\newcommand{\constraint}[1]{\ensuremath{(\prule{}{#1})}}
\newcommand{\lrule}[3]{\ensuremath{#1\mathrm{:}\;\;#2 & #3}}
\newcommand{\lprule}[3]{\ensuremath{#1\mathrm{:}\;\;\prule{#2}{#3}}}

\newcommand{\pair}[2]{\ensuremath{\langle{#1},{#2}\rangle}}
\newcommand{\T}[1]{\ensuremath{\mathbf{T}_{#1}}}
\newcommand{\F}[1]{\ensuremath{\mathcal{F}_{#1}}}
\newcommand{\AS}[1]{\ensuremath{\mathcal{J}_{#1}}}
\newcommand{\GAS}[1]{\ensuremath{\mathcal{J}^{\star}_{#1}}}
\newcommand{\naf}[1]{\ensuremath{\mathit{not}\;#1}}
\newcommand{\hbase}[1]{\ensuremath{\mathcal{B}_{#1}}}

%% Zaken van Jeroen
\newcommand{\fimp}{\ensuremath{\rightsquigarrow}}
\newcommand{\supr}{\ensuremath{\sup\limits}}
\newcommand{\infi}{\ensuremath{\inf\limits}}
\newcommand{\aspr}{\leftarrow} % ASP rule
\newcommand{\aspprog}[1]{$$\begin{array}{l@{\;\aspr\;}l}#1\end{array}$$}
\newcommand{\aspn}[1]{\textit{not}\;#1}
\newcommand{\hint}[1]{\lbrace\textnormal{#1}\rbrace}
\newcommand{\imp}{\Rightarrow}
\newcommand{\minof}[2]{\ensuremath{\min(#1,#2)}}

%% More structure:
\newcommand{\suproff}[1]{\ensuremath{\sup_{}\{{#1}\}}}
\newcommand{\infofset}[2]{\ensuremath{\inf_{#1}\{{#2}\}}}
\newcommand{\suprof}[2]{\ensuremath{\sup_{#1}({#2})}}
\newcommand{\infof}[2]{\ensuremath{\inf_{#1}({#2})}}
\newcommand{\fimplies}[2]{\ensuremath{{#1}\rightsquigarrow{#2}}}

\newcommand{\qed}{\hfill \ensuremath{\Box}}

\newtheorem{definition}{Definition}
\newtheorem{example}{Example}
\newtheorem{theorem}{Theorem}
\newtheorem{corollary}{Corollary}
\newenvironment{proof}{\textbf{Proof\\}}{\qed{}\vspace{1em}}
\newenvironment{Proof}{\textbf{Proof\\}}{\qed{}\vspace{1em}}

\title{Fuzzy Argumentation Frameworks}
\date{\today}
\author{
Jeroen Janssen\thanks{Funded by the Fund for Scientific Research - Flanders (FWO-Vlaanderen)}\\
  Dept. of Computer Science \\
  Vrije Universiteit Brussel\\
  jeroen.janssen@vub.ac.be \\
\and Martine De Cock \\
  Dept. of Appl. Math. \\
  and Computer Science\\
  Universiteit Gent \\
  martine.decock@ugent.be
\and Dirk Vermeir \\
   Dept. of Computer Science \\
   Vrije Universiteit Brussel\\
   dvermeir@vub.ac.be \\
}

\maketitle
\bibliographystyle{alpha} 

\begin{abstract}
We propose fuzzy argumentation frameworks as a conservative extension
of traditional argumentation frameworks\cite{dung95}. The fuzzy
approach enriches the expressive power of the classical argumentation
model by allowing to represent the relative strength of the attack
relationships between arguments, as well as the degree to which
arguments are accepted. Furthermore, we explore the relationship with
fuzzy answer set programming, more in particular the correspondence
between the stable extensions of a fuzzy argumentation framework and
the fuzzy models of an associated program.

{\bf Keywords:} Fuzzy argumentation frameworks, fuzzy answer set programming, query expansion.
\end{abstract}

%%%%%%%%%%%%%%%%%%%%%%%%%%%%%%%%%%%%%%%%%%%%%%%%%%%%%%%%%%%%%%%%%%%%%%%%%%%%%%%%%%%%%
%
%      1. Introduction
%
%%%%%%%%%%%%%%%%%%%%%%%%%%%%%%%%%%%%%%%%%%%%%%%%%%%%%%%%%%%%%%%%%%%%%%%%%%%%%%%%%%%%%

\section{Introduction}\label{sec-intro}
The challenge of understanding argumentation and its role in human
reasoning has been addressed by many researchers in different fields,
including philosophy, logic, and AI.  A formal theory of argumentation
has been proposed in a seminal article by Dung\cite{dung95}. This
theory is based on the idea that a statement (argument) is believable
if it can be \emph{defended} successfully against contesting arguments.
Thus, the theory considers so-called \emph{argumentation frameworks} that
model the interactions between different arguments, while abstracting
from the meaning or internal structure of an argument.  Within a given
framework of interacting arguments, there might be one, or more, sets
of conclusions, called \emph{extensions}, that are deemed collectively
acceptable. 

In human argumentation however, in many cases, not all attacks have equal strength.
For example, in a murder trial, the argument ``her fingerprints are on
the murder weapon'' is a stronger attack on the premise ``the accused is
innocent'' than the argument that ``she was the last person to see the
victim alive''. Furthermore, statements like ``the accused is innocent'' and ``the accused did not have a good relationship with the victim'' may hold to, and hence may be deemed acceptable to, a certain extent only, as opposed to being either accepted or not.
In the present paper, we therefore introduce \emph{fuzzy argumentation
frameworks} as a conservative extension of \cite{dung95}'s notion.
Advantages of the fuzzy approach include the possibility to
represent the relative strength of the (attack) relationships
between arguments and a more sophisticated approach to
extensions: being fuzzy sets of arguments, an extension may
commit only to a certain degree to the acceptance of a certain
argument, thus weakening any (unanswered) attacks on it. 

Already in \cite{dung95}, a strong link between non-monotonic
reasoning formalisms and argumentation frameworks was presented. E.g.
the stable model semantics for logic programs naturally maps to
argumentation frameworks where an argument constitutes a derivation in
the program. In the present paper, we take a different approach: we
show that a program can be obtained from a fuzzy argumentation framework such that the
fuzzy models of the program correspond to the stable extensions
of the framework.

The rest of the paper is organised as follows: to introduce the notation used in the paper, in Section~\ref{sec-prelim} we recall some preliminaries re. fuzzy logic. Next we introduce fuzzy argumentation frameworks by providing the basic definitions and properties (Section~\ref{sec-faf}) as well as a sample application (Section~\ref{sec-appl}). The relationship with fuzzy answer set programs is
established in Section~\ref{sec-fasp} and Section~\ref{sec-conclusions} concludes.

Due to space restrictions, all proofs have been omitted. They can be
obtained from~\url{http://tinf2.vub.ac.be/~dvermeir/papers/faf-full.pdf}.


%%%%%%%%%%%%%%%%%%%%%%%%%%%%%%%%%%%%%%%%%%%%%%%%%%%%%%%%%%%%%%%%%%%%%%%%%%%%%%%%%%%%%
%
%      2. Preliminaries
%
%%%%%%%%%%%%%%%%%%%%%%%%%%%%%%%%%%%%%%%%%%%%%%%%%%%%%%%%%%%%%%%%%%%%%%%%%%%%%%%%%%%%%

\section{Preliminaries}\label{sec-prelim}

Throughout this paper, membership values are taken from a complete lattice called a  \emph{truth lattice}, i.e.~a partially
ordered set $(\Lattice, \leq_\Lattice)$ where every subset of
\Lattice{} has a greatest lower bound ($\inf$) and a least upper bound
($\sup$)\cite{birkhoff:1967}. Where the order is clear from the context, we 
refer to the lattice as $\Lattice$ instead of $(\Lattice, \leq_\Lattice)$. In addition we use $0_\Lattice$ and $1_\Lattice$,
or $0$ and $1$ if \Lattice{} is understood,
to denote the smallest and greatest element of \Lattice{},
respectively. 
The traditional logical operations of negation, conjunction, disjunction, and implication
are generalized to logical operators acting on
truth values of \Lattice{} in the usual way (see e.g.~\cite{novak:1999}):
\begin{itemize}
\item A \emph{negator} on \Lattice{} is any anti-monotone $\Lattice \to \Lattice$
mapping \fneg{} satisfying
$\fneg 0_\Lattice = 1_\Lattice$ and $\fneg 1_\Lattice = 0_\Lattice$. 
A negator
\fneg{} is called \emph{involutive} iff \Forall{x\in\Lattice}{\fneg \fneg x = x}.
\item
A \emph{triangular norm}, t-norm for short, on $\Lattice$ is any
commutative and associative $\Lattice^2 \to \Lattice$ (infix) operator $\fand$ satisfying
\Forall{x\in\Lattice}{1_\Lattice \fand x = x}. Moreover we require $\fand$ to be
increasing in both of its arguments, i.e.\footnote{Note that the
monotonicity of the second component immediately follows from that of the
first component due to the commutativity.} for $x_1, x_2 \in \Lattice$,
$x_1 \leq_{\Lattice} x_2$ implies $x_1 \fand y \leq_{\Lattice} x_2 \fand y$.
Intuitively, a t-norm corresponds to conjunction.
\item
A \emph{triangular conorm}, t-conorm for short, on $\Lattice$ is any
commutative and associative $\Lattice^2 \to \Lattice$
(infix) operator $\for$ satisfying \Forall{x\in\Lattice}{0_\Lattice \for x = x}.
Moreover we require $\for$ to be increasing in both of its arguments. 
A t-conorm corresponds to disjunction.
\item
An \emph{implicator} $\fimp$ on $\Lattice$ is any $\Lattice^2 \to \Lattice$ (infix) operator
$\fimp$ satisfying $\fimplies{0_\Lattice}{0_\Lattice} = 1_\Lattice$, and 
\Forall{x\in\Lattice}{\fimplies{1_\Lattice}{x} =  x}.
Moreover $\fimp$ must be decreasing in its first, and
increasing in its second argument, i.e., for $x_1,x_2,y\in\Lattice$,~$x_1 \leq_{\Lattice} x_2$
implies $\fimplies{x_1}{y} \geq_{\Lattice} \fimplies{x_2}{y}$ as well as 
$\fimplies{y}{x_1} \leq_{\Lattice} \fimplies{y}{x_2}$.
\end{itemize}
The \emph{residual implicator} of a t-norm $\fand$ is defined by 
$\fimplies{x}{y} = \sup\set{\lambda\in\Lattice{}\mid x\fand \lambda
  \leq_\Lattice{} y}$,
while a t-conorm $\for$ and a negator $\fneg$ induce an S-implicator defined by 
$\fimplies{x}{y} = \fneg x \curlyvee y$.

In this paper, we will mostly assume that truth lattices are finite.
E.g. finite subsets of $[0,1]$ such as \set{0.0, 0.1, \ldots, 0.9,
1.0} are used frequently. Well-known fuzzy logical operators include
the \emph{minimum t-norm} $x \fand y = \min(x,y)$, its residual implicator $x
\fimp y = 1$ if $x \leq y$, and $x \fimp y = y$ otherwise, as well as
the corresponding S-implicator $x \fimp y = \max(1-x,y)$. The
\emph{{\L}ukasiewicz t-norm} $x \fand y = \max(x+y-1,0)$ induces the residual
implicator $x \fimp y = \min(1-x+y,1)$, which is also an S-implicator.
For negation, often the \emph{standard negator} $\fneg x = 1-x$ is used. 

A \emph{fuzzy set} $A$ over some (ordinary) set $X$ and a truth lattice
\Lattice{} is an $X \rightarrow \Lattice{}$ mapping. We use \Fuzzy{X},
where \Lattice{} is understood,
to denote the set of all fuzzy sets over $X$.
We sometimes use the notation $x^l$ to denote that $A(x) = l$.
The \emph{support} of a fuzzy set $A$ is defined by
$\ssupport{A} = \set{x \mid A(x) > 0_\Lattice}$.
Fuzzy set intersection is defined by $(A \cap B)(x) = A(x) \fand B(x)$. Fuzzy set inclusion is also defined as usual by $A \subseteq_{\Lattice} B$, or $A \subseteq B$ if \Lattice{} is understood, iff \Forall{x\in X}{A(x) \leq_{\Lattice} B(x)}.
A fuzzy relation over $X$ is a fuzzy set over $X \times X$.

%We will also use the notion of \emph{full image} and \emph{subdirect
%image} of a fuzzy set
%under a fuzzy relation (see e.g. \cite{bocoke:2003}): for
%a fuzzy set $X$ over $U$ and a binary fuzzy binary relation $R$
%over $U\times U$, the full image of $X$ under $R$ is a fuzzy set
%\fullimg{X}{R}over $U$ defined by
%\[
%\fullimg{X}{R} (x) = \suprof{y\in U}{ X(y) \fand R(y,x)}
%\]
%while the subdirect image of $X$ under $R$ is defined by
%\[
%\subdirectimg{X}{R} (y) = \infof{x\in U}{\fimplies{X(x)}{R(x,y)}}
%\]

%%%%%%%%%%%%%%%%%%%%%%%%%%%%%%%%%%%%%%%%%%%%%%%%%%%%%%%%%%%%%%%%%%%%%%%%%%%%%%%%%%%%%
%
%      3. Fuzzy argumentation frameworks
%
%%%%%%%%%%%%%%%%%%%%%%%%%%%%%%%%%%%%%%%%%%%%%%%%%%%%%%%%%%%%%%%%%%%%%%%%%%%%%%%%%%%%%

\section{Fuzzy Argumentation Frameworks}\label{sec-faf}

An argumentation framework\cite{dung95} consists of a set of
\emph{arguments}, some of which \emph{attack} each other. Intuitively,
an \emph{extension} of such a framework represents a position, i.e.
a set of arguments a rational agent may subscribe to, that
can be defended against attacks. 

\begin{definition}\label{def-af}
An \emph{argumentation framework} (\abaf{} for short) is a tuple
\af{\mc{A}}{\attack{}}
where $\mc{A}$ is a set of arguments and 
$\attack{} \subseteq \mc{A} \times \mc{A}$
is a binary \emph{attack} relation between arguments. When $a_1 \attack a_2$,
we say that the argument $a_1$ \emph{attacks} the argument
$a_2$. The notation is extended to sets in the obvious way: i.e.~for $b$ an argument, and $A$ and $B$ sets of arguments, 
$A \attack b$ iff
\Exists{a\in A}{a\attack b}, while
$b \attack A$ iff \Exists{a\in A}{b\attack a}, and
$A \attack B$ iff \Exists{b\in B}{A\attack b}.
\end{definition}

\begin{definition}\label{def-afb}
Let \af{\mc{A}}{\attack{}} be an \abaf{}. %argumentation framework.
A set $A\subseteq \mc{A}$ is \emph{conflict-free} if
no argument in $A$ attacks an argument in $A$,
i.e. $\neg (A\attack A)$.
A conflict-free set $A$ is an \emph{admissible extension} if it defends
itself against all attacks, i.e. \Forall{b\attack A}{A\attack b}.
A \emph{preferred extension} is a maximal (w.r.t. $\subseteq$) admissible extension.
A \emph{stable extension} is a preferred extension $A$ that attacks all
external arguments, i.e. \Forall{b\not\in A}{A\attack b}.
\end{definition}

\begin{example}\cite{gorkar96}\label{exaf1}
The \abaf{}
\af{\set{\mathit{porsche},\mathit{volvo},\mathit{safe},\mathit{sporty}}}{R},
 where $R$ contains
$\mathit{porsche}\attack\mathit{volvo}$,
$\mathit{volvo}\attack\mathit{porsche}$,
$\mathit{safe}\attack\mathit{sporty}$,
$\mathit{safe}\attack\mathit{porsche}$,
$\mathit{sporty}\attack\mathit{safe} $, and
$\mathit{sporty}\attack\mathit{volvo}$,
represents a discussion between a wife and her
husband about buying a car. There are two stable extensions:
\set{\mathit{sporty},\mathit{porsche}}
and
\set{\mathit{safe},\mathit{volvo}}. Note that, on its own,
e.g.~\set{\mathit{safe}} is already admissible (but not preferred).
\end{example}

\begin{example}\label{exfaf0}
The \abaf{} 
\af{\set{\mathit{red},\mathit{pink},\mathit{blue}}}{R},
where $R$ contains
$\mathit{red}\attack\mathit{pink}$,
$\mathit{pink}\attack\mathit{red}$,
$\mathit{red}\attack\mathit{blue}$,
$\mathit{blue}\attack\mathit{red}$,
$\mathit{pink}\attack\mathit{blue} $, and
$\mathit{blue}\attack\mathit{pink}$,
represents a discussion about the colour of a certain sweater in the context of a school uniform policy. There are three stable extensions:
\set{\mathit{red}},
\set{\mathit{pink}},
and
\set{\mathit{blue}}. 
\end{example}
All three arguments in Example \ref{exfaf0} are conflicting. However,
the argument that the sweater is red (e.g.~as raised by a teacher) and
the argument that the sweater is pink (e.g.~as claimed by the school
principal) are not conflicting to a very high
degree, because for a light shade of red one person might call it pink,
while another might still prefer to call it red.  Suppose that the
school policy demands for a blue uniform. Both the teacher's and the
school principal's argument strongly attack the argument that the
sweater conforms to the school policy (e.g.~as claimed by the
student). To harvest the best arguments to attack the student's
argument, it is therefore desirable to be able to include both the
teacher's and the school principal's argument in the extension to a high degree.
Indeed the latter two attack each other only to a small
degree, hence, intuitively, committing to them both to a relatively
high degree does not significantly violate the conflict-freeness of the extension. 

To allow this kind of expressivity, we extend the classical argumentation model in two ways: (1) by allowing the attack relation to be a fuzzy relation over the set of arguments, we can represent the degree to which arguments attack each other, and (2) by allowing an extension to be a fuzzy set over the set of arguments, we can model that some arguments are accepted to a higher degree than others. 
%Also, one would expect it to be possible that an agent accepts an argument ``to a
%certain degree'', or even to weakly accept contradictory arguments,
%e.g.  to express doubt.  Hence, in the following definition, an
%\emph{extension} will be a fuzzy set of accepted arguments.
In the following definition, we formalize these intuitions by
extending \attack{} to cover fuzzy sets of arguments. A fuzzy argumentation framework consists of a set of arguments, some of which {\em attack each other to a certain degree}.

\begin{definition}\label{def-faf}
A \emph{fuzzy argumentation framework} (\abfaf{} for short) is a tuple
$\af{\mc{A}}{\attack}$ where \mc{A} is a set of arguments and \attack{}
is a fuzzy relation over $\mc{A}$.
%An \emph{extension} of $F$ is a fuzzy set $A\in\Fuzzy{\mc{A}}$
%over some truth lattice \Lattice{}.
For $b$ an argument, and $A$ and $B$ fuzzy sets of arguments, we define the degree to which $A$ attacks $b$ as 
$A\attack{}b$ $=$ $\suprof{a\in\mc{A}}{A(a) \fand (a\attack{}b)}$,
and the degree to which $b$ attacks $A$ as
$b\attack{}A$ $=$ $\suprof{a\in\mc{A}}{A(a) \fand (b\attack{}a)}$.
%For $A,B\in\Fuzzy{\mc{A}}$ and $B\subseteq\mc{A}$, we define
Furthermore, the degree to which $A$ attacks $B$ is given by
$A\attack{}B$ $=$ $\suprof{b\in\mc{A}}{B(b) \fand (A\attack{}b)}$.
\end{definition}
According to Definition \ref{def-faf}, the strength of an attack $A\attack{}b$ does not only depend on the strength of an attack $a\attack{}b$, where $a$ is an argument supported by $A$, but also on the degree $A(a)$ to which $A$ supports $a$: the stronger the presence of $a$ in $A$, the stronger the attacks from $A$ through $a$. On the other hand, if an argument $b$ is only present to a marginal degree in $B$, it should be clear that attacking $b$ does not greatly contribute to the ``global'' attack on $B$. 


\begin{example}\label{fuzzy-doc}
Let \af{\mc{A}}{\attack} be a \abfaf{} where \mc{A} contains
all arguments that appear in two position papers $A$ and $B$.
Since a position paper does not support each of its arguments in
equal measure, the papers are best represented by fuzzy sets
$A,B\in\Fuzzy{\mc{A}}$. Now suppose that there are two arguments
$a,b\in\mc{A}$ such that $a\attack{}^{0.9} b$, i.e. $a$ strongly
attacks $b$. On the other hand, $a$ is supported by $A$ but
not by $B$ while $b$ is not supported by $A$ and only weakly
by $B$, i.e. $b$ represents only a minor aspect of $B$. 
E.g. $A(a) = 0.9$, $B(a) = 0$, while $A(b) = 0$ and $B(b) = 0.1$.
\par
Then, although in supporting $a$, $A$ attacks $b$ strongly, i.e.~(using the minimum as t-norm in Definition~\ref{def-faf}) $A \attack{}^{0.9} b$, this attack does not noticeably affect $B$'s position as a whole since it contributes only $0.1$ to $A\attack{} B$.
\end{example}

Intuitively, an extension of a fuzzy argumentation framework is a fuzzy set $A$ such that for each $a$ in $\mc{A}$, $A(a)$ represents the degree to which a rational agent accepts argument $a$. Definition \ref{def-faf-1} provides more flexibility than Definition \ref{def-afb} in three aspects: (1) while in classical argumentation frameworks extensions are required to be entirely conflict-free, in the fuzzy approach they are allowed to contain {\em minor internal attacks}, (2) an admissible extension of a fuzzy argumentation framework only needs to {\em defend itself well enough against all attacks}, and (3), likewise, a stable extension only needs to {\em sufficiently attack all external arguments}, in other words, any argument that is to a high degree outside of the extension should be strongly attacked by it.

\begin{definition}\label{def-faf-1}
Let \af{\mc{A}}{\attack} be a \abfaf{}.
A fuzzy set $A$  over $\mc{A}$ is
\emph{$x$-conflict-free}, $x\in\Lattice{}$, iff $(\fneg (A \attack A)) \geq x$.
A fuzzy set $A$ is a \emph{$y$-admissible extension}
if it defends itself well enough against all attacks, i.e. 
$\infof{b\in\mc{A}}{(b \attack A) \fimp (A \attack b)} \geq y$.
A \emph{$y$-preferred extension}, $y \in \Lattice{}$, is a maximal (w.r.t. $\subseteq$ 
over fuzzy sets) $y$-admissible extension.  
A \emph{$z$-stable extension}, $z\in\Lattice{}$, is a fuzzy set $A$ that sufficiently attacks all external arguments, i.e.
$\infof{b\in\mc{A}}{\fimplies{\fneg A(b)}{(A \attack b)}} \geq z$.
\end{definition}
%
%Note that $A \attack b = \fullimg{A}{\attack}(b)$
%and $a \attack B = \fullimg{B}{\attack^{-1}}(a)$.
%
%A $z$-stable extension $A$ is such that any element that is to a
%high degree outside of $A$, is strongly attacked by $A$. 
%

%The fuzzy version of conflict-free sets allows extensions to
%contain internal attacks. However, for an $x$-conflict-free set $A$,
%the strength of such attacks is bounded from above by $\fneg x$. 

\begin{example}\label{ex-sweater}
%Suppose we are discussing school uniform policy and arguments are
%exchanged about a certain sweater.  One person claims that the
%sweater is red (argument $a$), whilst the other claims it
%is pink (argument $b$).  Obviously, these are conflicting
%arguments and they attack each other to a certain degree.  But it is
%also clear that the arguments are not conflicting to a very high
%degree, because for a light shade of red one person calls it pink,
%while another still prefers to call it red.  So, we could say that 
%$a \attack^{0.1} b$ and $b\attack^{0.1} a$, thus expressing
%doubts about the true colour of the sweater.  Suppose now that the main
%contested argument is whether the sweater conforms to the school uniform
%policy and call this $c$. Then, if the school uniform is supposed to
%be blue, surely we can say that $a \attack^{1} c$ and $b\attack^{1}c$.
%When we want to harvest the best arguments to attack $c$, we
%would like both $a$ and $b$ to be included in our extension to a
%rather high degree, even if they attack each other to a small ($0.1$)
%degree.  Since whatever our doubts about which one
%of $a$ and $b$ is more true might be, we can surely conclude that if
%$a$ and $b$ have a rather high value, the sweater will not be blue.
%\par
Assume that we replace the \abaf{} in Example \ref{exfaf0} by a \abfaf{} where the fuzzy attack relation is given by
$\mathit{red}\attack^{0.1}\mathit{pink}$,
$\mathit{pink}\attack^{0.1}\mathit{red}$,
$\mathit{red}\attack^1\mathit{blue}$,
$\mathit{blue}\attack^1\mathit{red}$,
$\mathit{pink}\attack^1\mathit{blue} $, and
$\mathit{blue}\attack^1\mathit{pink}$.
We consider the fuzzy set $A = \set{\mathit{pink}^{0.8}, \mathit{red}^{0.7}}$, in other words we accept that the sweater has a colour between pink and red. 

To evaluate the conflict-freeness of $A$, we determine the degree to which there are internal attacks in $A$. Using the minimum t-norm we obtain that (cfr.~Definition \ref{def-faf}) $A \attack pink = 0.1$ and $A \attack red = 0.1$, hence $A \attack A = 0.1$, which indicates a minor level of conflict among the accepted arguments. Using the standard negator we obtain $\fneg (A \attack A) = 0.9$, hence $A$ is $0.9$-conflict-free. Note that when using the {\L}ukasiewicz t-norm, which is inherently more tolerant to minor inconsistencies, we would obtain that $A \attack pink = 0$ and $A \attack red = 0$, hence that $A$ is 1-conflict-free. Because of limited space, in the rest of this example we only consider the {\L}ukasiewicz t-norm and its residual implicator.

Next we verify how well $A$ defends itself against all attacks. Since $pink \attack A = 0$ and $red \attack A = 0$, the only real attack on $A$ comes from the argument that the sweater is blue, namely $blue \attack A = 0.8$. However, from $A \attack blue = 0.8$ and the previously mentioned attack values, it is clear that $A$ strikes back with equal force on all attacking arguments. Since for any residual implicator $a \attack A \fimp A \attack a$ equals 1 as soon as $a \attack A \leq A \attack a$, we obtain that $A$ is 1-admissible. 
% 
%Furthermore, with the Lukasiewicz t-norm and its residual implicator, we can calculate that $A \attack pink = 0$, $A \attack red = 0$ and $A \attack blue = 1$, as well as $blue \attack a = 1$, due to which $\infof{a \in \mc{A}}{a \attack A \fimp A \attack a} = 1$ and hence $A$ is $1$-admissible.  
It is however not a $1$-preferred set, since $\set{pink^{0.8},red^{0.8}}$ is a superset of $A$ which is also $1$-admissible.

As for the stability, $\fneg A(pink) \fimp A \attack pink = 0.2 \fimp 0 = 0.8$, $\fneg A(red) \fimp A \attack red = 0.3 \fimp 0 = 0.7$ and $\fneg A(blue) \fimp A \attack blue = 1 \fimp 0.8 = 0.8$.  By which we can conclude that $A$ is a $0.7$-stable extension. Note that $A$ does not have a perfect score for stability because, among other things, $A$ neither fully supports nor opposes the argument that the sweater is red.
\end{example}

It is straightforward to verify that \abfaf{}s represent a conservative extension
of the classical notion.

\begin{theorem}\label{faf-extends-af}
Let $F = \af{\mc{A}}{\attack{}}$ be a \abaf{} and let
$F_f = \af{\mc{A}}{\attack_f}$ be the fuzzy version over \set{0,1}
of $F$ with $a \attack_f^1 b$ iff $a\attack b$.
Then the set of stable extensions of $F$ coincides with the set of
supports of the $1$-conflict-free $1$-stable extensions of
$F_f$.
\end{theorem}

The following theorems describe some
of the effects of accepting more or less arguments (or accepting them
to a higher or a lower degree).  
The extended attack relation is
monotone w.r.t.~fuzzy set inclusion, i.e.~accepting more arguments
does not decrease the power to attack.

\begin{theorem}\label{MonotonStable1}
Let \af{\mc{A}}{\attack} be a \abfaf{} and $B\subseteq A \in \Fuzzy{\mc{A}}$.
Then $\Forall{a\in\mc{A}}{B \attack a \leq A \attack a}$.
\end{theorem}
\begin{proof}
 Let $A,B \in \mc{F}(\mc{A})$ such that $B \subseteq A$ and let $a \in \mc{A}$, 
 then if $B \attack a$ this means, by definition, that 
 $\suprof{b \in \mc{A}}{B(b) \fand (b \attack a)}$.  
 Now, due to the fact that $\Forall{a \in \mc{A}}{B(a) \leq A(a)}$ 
 and the monotonicity of $\for$ and $\fand$ it follows that 
 $\suprof{b \in \mc{A}}{B(b) \fand (b \attack a)} \leq 
 \suprof{b \in \mc{A}}{A(b) \fand (b \attack a)}$, 
 thus $B \attack a \leq A \attack a$
\end{proof}

Furthermore, $z$-stability is
monotone with respect to fuzzy set inclusion. This might come as a
surprise since it is not true in the boolean case, but it is due to
the fact that $z$-stability is independent of the degree of
conflict-freeness.


\begin{theorem}\label{stable-is-monotone}
Let \af{\mc{A}}{\attack} be a \abfaf{} and $B\subseteq A \in \Fuzzy{\mc{A}}$.
Then if $B$ is a $z$-stable extension, so is $A$.
\end{theorem}
\begin{proof}
Let $B\in\Fuzzy{\mc{A}}$ be $z$-stable and $B \subseteq
A\in\Fuzzy{\mc{A}}$. Due to the definition of $z$-stable,
this means that $\infof{a \in \mc{A}}{\fneg B(a) \fimp B \attack a} \geq z$.
>From this it follows that $\infof{a \in \mc{A}}{\fneg B(a) \fimp A \attack a} \geq z$
because of Theorem~\ref{MonotonStable1} and the monotonicity of $\fimp$ in its second argument.
Due to the fact that $\fneg$ is monotonically decreasing, that 
$\Forall{a \in \mc{A}}{B(a) \leq A(a)}$ and
that $\fimp$ is anti-monotone in its first argument, 
we obtain $\infof{a \in \mc{A}}{\fneg A(a) \fimp A \attack a} \geq z$
and thus $A$ is also $z$-stable.
\end{proof}

For $x$-conflict-freeness, we obtain
anti-monotonicity: adding more arguments, results in more conflicts,
so conflict-freeness decreases.

\begin{theorem}\label{AntiMonConflictFree}
Let \af{\mc{A}}{\attack} be a \abfaf{} and let $B\subseteq A \in \Fuzzy{\mc{A}}$.
Then if $A$ is $x$-conflict-free, so is $B$.
\end{theorem}
\begin{proof}
Let \af{\mc{A}}{\attack} be a \abfaf{} and let $B\subseteq A \in
\Fuzzy{\mc{A}}$ with $A$ $x$-conflict-free.
Then, $(B\attack B) \leq (A \attack A)$ follows from $B\subseteq A$,\\
$B\attack B=\suprof{a\in\mc{A}}{B(a) \fand \suprof{b\in\mc{A}}{B(b) \fand b \attack a}}$,\\
$A\attack A=\suprof{a\in\mc{A}}{A(a) \fand \suprof{b\in\mc{A}}{A(b) \fand b \attack a}}$,
and the monotonicity of~$\fand$ and~$\sup$.
\end{proof}

%%%%%%%%%%%%%%%%%%%%%%%%%%%%%%%%%%%%%%%%%%%%%%%%%%%%%%%%%%%%%%%%%%%%%%%%%%%%%%%%%%%%%
%
%      4. Query expansion
%
%%%%%%%%%%%%%%%%%%%%%%%%%%%%%%%%%%%%%%%%%%%%%%%%%%%%%%%%%%%%%%%%%%%%%%%%%%%%%%%%%%%%%

\section{Query Expansion}\label{sec-appl}
Query expansion is the process of expanding keyword based queries with
more terms, related to the intended meaning of the query, to refine
the search results. One option is to use an available thesaurus such
as WordNet, expanding the query by adding synonyms\cite{voorhees94}.
Related terms can also be automatically discovered from the searchable documents though, 
taking into account statistical information such as co-occurrences of words
in documents: the more frequently two terms co-occur, 
the more they are assumed to be related\cite{Xu96}.

%Suppose we have a thesaurus of terms, which relates terms that are
%similar to each other.
%Then we could just add all terms that are related by the thesaurus to
%one of the terms in the query.
%This can however lead to a query that contains too much information, especially
%when there are ambiguous terms in the query.
%In the example above, the term ``\emph{apple}'' could be related to
%both ``\emph{macintosh}'' as well as ``\emph{fruit}'',
%leading to an expansion with both terms, clearly not a good refinement.

In either case, the thesaurus can be thought of as being a (fuzzy) relation $R$ over the set $X$ of terms. Note that $R$ should be reflexive (i.e.~every term is obviously related to itself) and symmetric (i.e.~if we say that term $a$ is related to term $b$ to a degree of $p$, obviously we want that term $b$ is related to term $a$ in the same degree). Transitivity is not necessarily required, for reasons explained in \cite{FuzzyRoughSets2}.  
From $X$ and $R$, a \abfaf{} $F$ can be generated such that the conflict-free and stable extensions of $F$ correspond to optimal queries.  

More in particular, we generate the \abfaf{} $F = \af{X}{\attack}$ where $a \attack b = \fneg (a\,R\,b)$. In other words, the arguments of $F$ correspond to terms, and a term $a$ attacks a term $b$ to the extent to which these terms are not related. Note that, due to the symmetry of $R$, every argument in $X$ attacks each one of its attackers to the same degree. 

A set $A \subseteq X$ is then highly conflict-free if every term in $A$ is related to all other terms in $A$ to a high degree, in other words $A$ is a coherent query. Furthermore, $A$ is highly stable if it strongly attacks all external arguments, i.e.~a term can only be left out of the query to the extent that it is not related to at least one of the terms in the query. 
Together these requirements fit our intuition about what an optimally expanded query should look like, i.e.~a query that has a high stability (refined with as many terms as possible) and is highly conflict-free (without losing the coherence). The following example illustrates the interplay between the conflict-freeness and the stability requirements.

%\par
%A better approach takes the \emph{context} of the query in consideration
%and preferably uses a fuzzy thesaurus, so as to be able to correlate
%terms
%to a certain degree, i.e. ``\emph{apple}'' is correlated to
%``\emph{fruit}'' but is not a synonym of ``\emph{fruit}'', so the degree
%of correlation should not be $1$, but a rather high value.  


\begin{example}\label{vbqueryexp1}
\renewcommand{\tabcolsep}{0.2em}
\begin{table}[htb]
\caption{Fuzzy Thesaurus from \cite{FuzzyRoughSets2}\label{thesaurus}}
\centering
{\footnotesize
\begin{tabular}{|l|r|r|r|r|r|r|}
    \hline
&mac&recipe&computer&apple&fruit&pie\\
    \hline
mac&  $1$  &  $0$ &   $0.89$ & $0.89$ & $0$ & $0.01$\\
recipe&    &  $1$ &   $0.56$ & $0.83$ & $0.66$ & $1$\\
computer&  &      &   $1$    & $0.94$  & $0.44$ & $0.44$\\
apple &     &      &          & $1$     & $0.83$ & $0.99$\\
fruit &     &      &          &         & $1$ & $0.44$\\
pie   &     &      &          &         &     & $1$ \\
    \hline
\end{tabular}
}
\end{table}


\begin{table}[htb]
\caption{Attack Relation}\label{attack-relation}
\centering
{\footnotesize
\begin{tabular}{|l|r|r|r|r|r|r|}
    \hline
&mac&recipe&computer&apple&fruit&pie\\
    \hline
mac&  $0$  &  $1$ &   $0.11$ & $0.11$ & $1$ & $0.99$\\
recipe&    &  $0$ &   $0.44$ & $0.17$ & $0.44$ & $0$\\
computer&  &      &   $0$    & $0.06$  & $0.66$ & $0.66$\\
apple &     &      &          & $0$     & $0.17$ & $0.01$\\
fruit &     &      &          &         & $0$ & $0.66$\\
pie   &     &      &          &         &     & $0$ \\
    \hline
\end{tabular}
}
\end{table}

Suppose we have a set of terms
$X = \set{apple, \mathit{fruit}, mac, pie, recipe, computer}$
and a fuzzy relation $R$ over $X$, defined as in Table~\ref{thesaurus}. The corresponding attack relation, using the standard negator, is depicted in Table \ref{attack-relation}. 
%\par
%We can now do two things: regard queries as crisp sets (so a term is either in the query or not) and therefore,
% use an argumentation framework in which only the attacks will have values that can range %over a lattice,
%or use queries that are fuzzy sets.  The latter is for example useful in combination with %a suggestion system that suggests
%one term at a time to expand with.  If the expanded queries
%``\emph{apple,pie,recipe}'' and ``\emph{apple,pie,fruit}'' would have
%the same stability and conflict-freeness, we could give the term that
%has the highest membership in the expanded query preference above the
%other term.
%\par
%In the first case (where a query is a crisp set), suppose we are
%looking for a $0.6$-stable extension of the query 
The user query that we aim to expand is $Q_1=\{apple, pie\}$. Independently of the choice of t-norm in Definition \ref{def-faf}, it holds that $Q_1 \attack mac = 0.99$, $Q_1 \attack recipe = 0.17$, $Q_1 \attack computer = 0.66$, $Q_1 \attack apple = 0.01$, $Q_1 \attack fruit = 0.66$ and $Q_1 \attack pie = 0.01$. 

The query $Q_1$ is highly conflict-free, namely to degree 0.99 when using the standard negator, since the strength of the attack $apple \attack pie$ is almost neglectable. On the other hand, $Q_1$ is only a $0.17$-stable extension (independently of the choice of negator and of the choice of implicator in Definition \ref{def-faf-1}). This unstability is due to the term recipe being excluded from the query without there being a good reason to. Indeed, recipe is not strongly attacked by any of the keywords from the query so, in terms of stability, there is no reason not to include it. 
 
However, it is not possible to add more terms to the query while keeping the original high level of conflict-freeness. We therefore lower our standards and look for expanded queries that are $0.8$-conflict-free. The only expanded query that satisfies this condition is $Q_2 = \{apple, pie, recipe\}$, which is a $0.66$-stable extension.
\end{example}

\begin{example}\label{vbqueryexp2}
Adding more terms to query $Q_2$ in Example \ref{vbqueryexp1} results in a significant drop in conflict-freeness. Indeed, the remaining terms mac, computer, and fruit are under an attack with a strength of at least 0.66 by at least one of the terms already in $Q_2$, so including yet another term in $Q_2$ would leave the conflict-freeness at most at 0.44. However, in the setting described in this section, it is also possible to consider weighted queries. For instance, one can verify that, using the minimum t-norm, for $Q_3 = \{apple^1, pie^1, recipe^1, fruit^{0.3}\}$ we obtain that
$Q_3 \attack mac = 1$, $Q_3 \attack recipe = 0.3$, $Q_3 \attack computer = 0.66$, $Q_3 \attack apple = 0.17$, $Q_3 \attack fruit = 0.66$ and $Q_3 \attack pie = 0.3$. Hence, $Q_3$ is $0.7$-conflict-free, and, using the residual or the S-implicator associated with the minimum t-norm, $Q_3$ is still a $0.66$-stable extension.
\end{example}

%%%%%%%%%%%%%%%%%%%%%%%%%%%%%%%%%%%%%%%%%%%%%%%%%%%%%%%%%%%%%%%%%%%%%%%%%%%%%%%%%%%%%
%
%      5. Link with FASP
%
%%%%%%%%%%%%%%%%%%%%%%%%%%%%%%%%%%%%%%%%%%%%%%%%%%%%%%%%%%%%%%%%%%%%%%%%%%%%%%%%%%%%%

\section{FAF and Fuzzy Answer Set Programming}\label{sec-fasp}

The answer set programming (ASP) paradigm\cite{BaralBook} has gained a lot of popularity in the last years, due to its truly declarative non-monotonic semantics. ASP and fuzzy logic can be be combined into the single framework of fuzzy answer set programming to increase the flexibility and hence the application potential of ASP.
A \emph{fuzzy answer set program}\cite{fasp0}, \abfasp{} for short, is
a finite set of rules\footnote{
  In the present paper we do not consider negative
  literals of the form $\neg a$, $a$ an atom. In terms
  of \cite{fasp0}, this makes every interpretation $1$-consistent.
  Also, we do not consider constraints (rules with empty head).}
of the form \prule{a}{\alpha} with $a$ an \emph{atom}
and $\alpha$ a set of \emph{literals}, %$\alpha_i$
each of which
is either an atom or of the form \naf{a}, $a$ an atom,
representing the ``negation as failure'' of $a$.

A \emph{fuzzy interpretation} of a \abfasp{} $P$ is a mapping 
$I: \hbase{P} \rightarrow \Lattice{}$ assigning a truth value
from the lattice \Lattice{} to each of the atoms appearing in $P$.
$I$ is extended to literals by defining $I(\naf{a}) = \fneg I(a)$
and to rules \lprule{r}{a}{\alpha} using 
$I(\alpha) = \fand_{l\in\alpha} I(l)$ and 
$I(r) = \fimplies{I(\alpha)}{I(a)}$. 

A fuzzy $y$-\emph{model}, $y\in\Lattice$, of $P$ is a fuzzy
interpretation $I$ that satisfies $\mc{A}_p (P,I) \geq y$
where $\mc{A}_p$ is a function that takes as input a program and an
interpretation, yielding a value denoting the degree in which
$I$ is a model of $P$. Naturally, $\mc{A}_p$ should be increasing
whenever the degrees of satisfaction $I(r)$ of the rules in $P$ are
increasing.

We show that, under certain conditions, the $y$-stable extensions
of a \abfaf{} $F=\af{\mc{A}}{\attack}$ correspond to the fuzzy $y$-models %\cite{fasp0}
of a \abfasp{} $\Pi_F$ that can be constructed from $F$ as follows: 
intuitively, 
for each argument $a\in\mc{A}$, $\Pi_F$ will contain
exactly one rule $r_a$ introducing $a$. The body of $r_a$ contains
one literal $\naf{b_a}$ for each $b\in\mc{A}$ such that $(b\attack
a)>0_\Lattice{}$. Each literal $b_a$ is itself defined through a
single rule\footnote{
  Although program rules in~\cite{fasp0} syntactically may not contain
  constants from \Lattice{}, the semantics in \cite{fasp0} easily supports
  such an extension using the ``fuzzy input literals'' mechanism
  (Section~4 in~\cite{fasp0}).
} \prule{b_a}{b, (b\attack a)}.

\begin{definition}
For a \abfaf{} $F=\af{\mc{A}}{\attack}$, the associated \abfasp{} $\Pi_F$
is defined by $\Pi_F = R_a \cup R_b$
where
\begin{eqnarray*}
R_a & = & \set{ \prule{a}{
                     \set{\naf{b_a} \mid (b \attack a > 0_\Lattice{})}
                    } \mid a\in\mc{A}} \\
R_b & = & \set{ \prule{b_a}{b, (b\attack a)} \mid (b\attack{}a)>0_\Lattice{} }
\end{eqnarray*}
Where we take $min$ as the t-norm for aggregating the body in $R_a$-rules and the t-norm
of the argumentation framework for the $R_b$-rules.\\
The aggregator $\mc{A}_{\Pi_F}$ is such that all rules from $R_b$ must
evaluate to $1_\Lattice{}$, while for other rules, the minimal
degree of satisfaction is taken, i.e.
\[
\mc{A}_{\Pi_F}(\Pi_F, I) = \left\{
  \begin{array}{l@{\quad\quad}l}
  \infof{r_a \in R_a}{I(r_a)} & \mbox{\small iff ($\alpha$)}\\
  0_\Lattice{} & \mbox{\small otherwise}
  \end{array}
  \right.
\]
where $(\alpha) \equiv \Forall{r\in R_b}{I(r) = 1_\Lattice{}}$.
\end{definition}
Note that any argument $a$ that is not attacked will have a corresponding fact 
rule \prule{a}{} in $R_a$.

We show that
$y$-stable extensions of $F$ correspond to certain fuzzy $y$-models of $\Pi_F$.

\begin{theorem}\label{ThmFafFasp}
 Let $F=\af{\mc{A}}{\attack}$ be a \abfaf{}.
If $\fimp$ is a contrapositive implicator,
 $x = y \imp x \fimp y = 1$,
 $\fneg$ is an involutive negator and
 $\fneg \suprof{x \in X}{F(x)} = \infof{x \in X}{\fneg F(x)}$,
then for any $y \in \mc{L}$,
$X$ is a $y$-stable extension of $F$
iff $X' = X \cup \set{ b_a^q \mid q = (X(b) \fand (b \attack a))}$
is a $y$-model of $\Pi_F$.
\end{theorem}
\begin{proof}
Suppose $X$ is a $y$-stable extension of $F$.  By the definition of $y$-stable extensions, this is equivalent to
 $$\infof{a\in\mc{A}}{\fneg X(a) \fimp (X \attack a)} \geq y$$
Using contraposition this is equivalent to
 $$\infof{a\in\mc{A}}{\fneg(X \attack a) \fimp X(a)} \geq y$$
Then, by the definition of $X \attack a$
 $$\infof{a\in\mc{A}}{\fneg\suprof{b\in\mc{A}}{X(b) \fand (b \attack a)} \fimp X(a)} \geq y$$
Which is equivalent to
 $$\infof{a\in\mc{A}}{\fneg\suprof{b\in\mc{A} \wedge (b \attack a) > 0_\mc{L}}{X(b) \fand (b \attack a)} \fimp X(a)} \geq y$$
Due to the generalised De Morgan properties we get
 $$\infof{a\in\mc{A}}{\infof{b\in\mc{A} \wedge (b \attack a) > 0_\mc{L}}{\fneg(X(b) \fand (b \attack a))} \fimp X(a)} \geq y$$
By definition of $X'$ this is equivalent to
 $$\infof{a\in\mc{A}}{\infof{b\in\mc{A} \wedge (b \attack a) > 0_\mc{L}}{\fneg X'(b_a)} \fimp X'(a)} \geq y$$
Which, due to the construction of $\Pi_F$ and the interpretation of rules is equivalent to
 $$\infof{r \in R_a}{X'(r)} \geq y$$
Which is equivalent to the definition of a $y$-model of $\Pi_F$ due to the fact that $\inf$ is the aggregator for rules in $R_a$ if all rules in $R_b$ are $1$.  The latter is established by the fact that $X'(b_a) = X'(b) \fand (b \attack a)$ and by the restriction on the chosen implicator.
\end{proof}

Note that the above equivalence concerns \abfasp{} \emph{models}, not
answer sets where answer sets are models that are free from
``assumptions''\cite{fasp0} and, in a sense, minimal models. 
However, unlike with answer set programming,
minimality is not a desirable criterion for argumentation frameworks
as, intuitively, one attempts to maximize the set of arguments that can be defended
against attacks. The only limit on the size of an extension is the
desire for conflict-freeness (see also
Theorems~\ref{stable-is-monotone} and~\ref{AntiMonConflictFree})
which can be imposed separately.

%%%%%%%%%%%%%%%%%%%%%%%%%%%%%%%%%%%%%%%%%%%%%%%%%%%%%%%%%%%%%%%%%%%%%%%%%%%%%%%%%%%%%
%
%      6. Conclusion
%
%%%%%%%%%%%%%%%%%%%%%%%%%%%%%%%%%%%%%%%%%%%%%%%%%%%%%%%%%%%%%%%%%%%%%%%%%%%%%%%%%%%%%

\section{Concluding Remarks}\label{sec-conclusions}
%and Directions for Further Research}
We've motivated and introduced fuzzy argumentation frameworks as a conservative
extension of the classical notion from~\cite{dung95} which allows for
more fine-grained knowledge representation in terms of doubts (some
degree of conflict may be tolerated) and strength of attacks, both absolute and
with respect to the degree of acceptance of the attacker and the
attacked. The strong connection with Fuzzy Answer Set
Programming\cite{fasp0} also provides a practical implementation
using \textsc{dlvhex}\cite{fasp1}. In future work, we intend to extend
our approach to (applications of) bipolar argumentation 
frameworks\cite{cayrol2005,karacapilidis2001,verheij2002,amgoud2004}.
We will also investigate dialogical semantics, see e.g. \cite{jv99.1}, 
for fuzzy argumentation frameworks.
\bibliography{bib}
\end{document}
